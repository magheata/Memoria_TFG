%!TeX root=MemoriaTFG.tex

\chapter{Introducción}\label{introduccion}
% \todo{Esto es una prueba. Isaac}

Esta es una breve introducción a los distintos capítulos que componen esta memoria. \\

En el capítulo 2 \textbf{se explicarán brevemente los conceptos más importantes} que se deben conocer para poder tener una base inicial antes de profundizar con el tema del trabajo. En concreto, se introducirá el concepto de inteligencia artificial, los tipos que existen, los enfoques abordados en su diseño, las redes neuronales artificiales y los distintos tipos de aprendizaje utilizados para su entreno. \\

En el capítulo 3 \textbf{se expondrá la definición del proyecto} desarrollado. \\

En el capítulo 4 \textbf{se realiza un análisis técnico del proyecto} en el que se incluyen los objetivos, las restricciones, los reqisitos funcionales y no funcionales y un diagrama temporal del desarrollo del proyecto. \\

En el capítulo 5 \textbf{se explicarán los conceptos técnicos} que se emplearán en la memoria. \\

En el capítulo 6 \textbf{se profundiza en la implementación del proyecto}. \\